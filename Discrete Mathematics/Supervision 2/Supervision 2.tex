\documentclass{article}

\usepackage[a4paper, margin=0.5in]{geometry}
\usepackage{amsmath}
\usepackage{float}
\usepackage{amssymb}

\title{Discrete Maths Supervision Work 2}
\author{Morgan Saville}

\newcommand{\incomplete}{\textbf{\\ \\ INCOMPLETE}}
\newcommand{\rtp}{\textbf{RTP: }}
\newcommand{\myexists}[1]{\exists #1 \:}
\newcommand{\myforall}[1]{\forall #1 \:}
\newcommand{\rem}[2]{\textrm{rem}(#1, #2)}
\newcommand{\case}[1]{\item [\textbf{C#1}:]$ $ \\}

\begin{document}
    \maketitle
    
    \section{2.2}
    \begin{enumerate}
        \item $(i, k, l, m) = (-1, 1, 6, 5)$ meets the requirement
        \item \rtp $\myforall{N} \myforall{k_0,k_1.k_2,...k_N} (\myexists{a} \sum_{i=0}^{N}k_i 10^{i}=3a \iff \myexists{b} \sum_{i=0}^{N}k_i=3b)$\\
            Let $N$ arbitrary natural number, and let $k_0,k_1,k_2,...,k_N$ arbitrary natural numbers. \\
            First we prove `$\Rightarrow$'. \\
            Assume $\myexists{a} \sum_{i=0}^{N}k_i 10^{i}=3a$\\
            Instantiating, let $a$ such that $\sum_{i=0}^{N}k_i 10^{i}=3a$\\
            \rtp $\myexists{b} \sum_{i=0}^{N}k_i = 3b$
            \incomplete
        \item \rtp $\myforall{n} (\rem{n^2 + 1}{4}=0\lor\rem{n^2 + 1}{4}=1)$
            \begin{enumerate}
                \case{0} $n=2k$ for some integer $k$ \\
                    $\rem{n^2}{4} = \rem{(2k)^2}{4} = \rem{4k^2}{4} = 0$
                \case{1} $n=2k+1$ for some integer $k$ \\
                    $\rem{n^2}{4} = \rem{(2k+1)^2}{4} = \rem{4k^2 + 4k + 1}{4} = \rem{4(k^2 + k) + 1}{4} = 1$
            \end{enumerate}
            Since the above cases are exhaustive, we have shown the required statement.
        \item $ $
            \begin{enumerate}
                \item $\rem{55^2}{79} = \rem{3025}{79} = 23$
                \item $\rem{23^2}{79} = \rem{529}{79} = 55$
                \item $\rem{23\cdot 55}{79} = \rem{1265}{79} = 1$
                \item \begin{align*}
                        \rem{55^{78}}{79} &= \rem{(55^2)^{39}}{79} \\
                        &= \rem{23^{39}}{79} \\
                        &= \rem{23\cdot (23^2)^{19}}{79} \\
                        &= \rem{23\cdot 55 \cdot (55^2)^9}{79} \\
                        &= \rem{23\cdot (23^2)^4}{79} \\
                        &= \rem{23\cdot 55^2 \cdot 55^2}{79} \\
                        &= \rem{23 \cdot 23 \cdot 23}{79} \\
                        &= \rem {55 \cdot 23}{79} \\
                        &= 1
                    \end{align*}
            \end{enumerate}
        \item \begin{align*}
                2^{153} &\equiv 2 \cdot (2^8)^{19} \\
                &\equiv 2 \cdot 256^{19} \\
                &\equiv 2 \cdot 103^{19} \\
                &\equiv 206 \cdot (103^2)^9 \\
                &\equiv 53 \cdot 10609^9 \\
                &\equiv 53 \cdot 52 ^ 9 \\
                &\equiv 2756 \cdot (52^2)^4 \\
                &\equiv 2 \cdot (103^2)^2 \\
                &\equiv 2 \cdot 52^2 \\
                &\equiv 2 \cdot 103 \\
                &\equiv 206 \\
                &\equiv 53 \pmod{153}
            \end{align*}
            This does not contradict Fermat's Little Theorem because 153 is not prime.
        \item $ $
            \begin{enumerate}
                \item $\mathbb{Z}_3$ \begin{table}[H]
                    \centering
                    \begin{tabular}{|c|c|c|c|c|c|}
                    \hline
                    $a$ & $b$ & $a+b$ & $ab$ & $-b$ & $\frac{1}{b}$ \\
                    \hline
                    0 & 0 & 0   & 0   & 0  &     \\
                    0 & 1 & 1   & 0   & 2  & 1   \\
                    0 & 2 & 2   & 0   & 1  & 2   \\
                    1 & 1 & 2   & 1   &    &     \\
                    1 & 2 & 0   & 2   &    &     \\
                    2 & 2 & 1   & 1   &    &     \\
                    \hline
                    \end{tabular}
                    \end{table}
                    \item $\mathbb{Z}_6$ \begin{table}[H]
                        \centering
                        \begin{tabular}{|c|c|c|c|c|c|}
                        \hline
                        $a$ & $b$ & $a+b$ & $ab$ & $-b$ & $\frac{1}{b}$ \\
                        \hline
                        0 & 0 & 0   & 0   & 0  &     \\
                        0 & 1 & 1   & 0   & 5  & 1   \\
                        0 & 2 & 2   & 0   & 4  &     \\
                        0 & 3 & 3   & 0   & 3  &     \\
                        0 & 4 & 4   & 0   & 2  &     \\
                        0 & 5 & 5   & 0   & 1  & 5   \\
                        1 & 1 & 2   & 1   &    &     \\
                        1 & 2 & 3   & 2   &    &     \\
                        1 & 3 & 4   & 3   &    &     \\
                        1 & 4 & 5   & 4   &    &     \\
                        1 & 5 & 0   & 5   &    &     \\
                        2 & 2 & 4   & 4   &    &     \\
                        2 & 3 & 5   & 0   &    &     \\
                        2 & 4 & 0   & 2   &    &     \\
                        2 & 5 & 1   & 4   &    &     \\
                        3 & 3 & 0   & 3   &    &     \\
                        3 & 4 & 1   & 0   &    &     \\
                        3 & 5 & 2   & 3   &    &     \\
                        4 & 4 & 2   & 4   &    &     \\
                        4 & 5 & 3   & 2   &    &     \\
                        5 & 5 & 4   & 1   &    &     \\
                        \hline
                        \end{tabular}
                        \end{table}
                        \item $\mathbb{Z}_7$ \begin{table}[H]
                            \centering
                            \begin{tabular}{|c|c|c|c|c|c|}
                            \hline
                            $a$ & $b$ & $a+b$ & $ab$ & $-b$ & $\frac{1}{b}$ \\
                            \hline
                            0 & 0 & 0   & 0   & 0  &     \\
                            0 & 1 & 1   & 0   & 6  & 1   \\
                            0 & 2 & 2   & 0   & 5  & 4   \\
                            0 & 3 & 3   & 0   & 4  & 5   \\
                            0 & 4 & 4   & 0   & 3  & 2   \\
                            0 & 5 & 5   & 0   & 2  & 3   \\
                            0 & 6 & 6   & 0   & 1  & 6   \\
                            1 & 1 & 2   & 1   &    &     \\
                            1 & 2 & 3   & 2   &    &     \\
                            1 & 3 & 4   & 3   &    &     \\
                            1 & 4 & 5   & 4   &    &     \\
                            1 & 5 & 6   & 5   &    &     \\
                            1 & 6 & 0   & 6   &    &     \\
                            2 & 2 & 4   & 4   &    &     \\
                            2 & 3 & 5   & 6   &    &     \\
                            2 & 4 & 6   & 1   &    &     \\
                            2 & 5 & 0   & 3   &    &     \\
                            2 & 6 & 1   & 5   &    &     \\
                            3 & 3 & 0   & 2   &    &     \\
                            3 & 4 & 0   & 5   &    &     \\
                            3 & 5 & 1   & 1   &    &     \\
                            3 & 6 & 2   & 4   &    &     \\
                            4 & 4 & 1   & 2   &    &     \\
                            4 & 5 & 2   & 6   &    &     \\
                            4 & 6 & 3   & 3   &    &     \\
                            5 & 5 & 3   & 4   &    &     \\
                            5 & 6 & 4   & 2   &    &     \\
                            6 & 6 & 5   & 1   &    &     \\
                            \hline
                            \end{tabular}
                            \end{table}    
            \end{enumerate}
        \item Assume $n^3 \equiv (\rem{n}{6})^3 \pmod{6}$. We can therefore check all possibilities for $\rem{n}{6}$
            \begin{table}[H]
                \centering
                \begin{tabular}{|c|c|c|}
                \hline
                $\rem{n}{6}$ & $(\rem{n}{6})^3$  & $\rem{(\rem{n}{6})^3}{6}$ \\
                \hline
                0 & 0   & 0 \\
                1 & 1   & 0 \\
                2 & 8   & 6 \\
                3 & 27  & 3 \\
                4 & 64  & 4 \\
                5 & 125 & 5 \\
                \hline
                \end{tabular}
                \end{table}
            Since $\rem{(\rem{n}{6})^3}{6} \equiv (\rem{n}{6})^3 \equiv n^3 \pmod{6}$, we can see that $\myforall{n} n^3 \equiv n \pmod{6}$
        \item Assume $n \equiv 1 \pmod{p-1}$. \\
            Equivalently, assume $n = j(p-1) + 1$ for some integer $j$ \\
            \rtp $\myforall{i} \textrm{not multiple of }p\;i^n \equiv i \pmod{p}$ \\
            By universal instantiation, let $i$ some positive integer not a multiple of $p$. \\
            \rtp $i^n \equiv i \pmod{p}$ \\
            Equivalently, \rtp $i^n = kp + i$ for some integer $k$. \\
            Substituting $n$ into the left-hand side,
            \begin{align*}
                i^{j(p-1) + 1} &\equiv i^{jp+ (1-j)} \\
                &\equiv (i^p)^j \cdot i^{1-j} \\
                &\equiv i^j \cdot i^(1-j) \textrm{\; by Fermat's Little Theorem} \\
                &\equiv i^1 \\
                &\equiv i \pmod{p}
            \end{align*}
            As required.
        \item $n^7 \equiv n \pmod 7 \textrm{\; By question 8}$ \\
            $n^7 \equiv n^3n^3n \equiv n\cdot n\cdot n \equiv n^3 \equiv n \pmod 6 \textrm{\; By question 7}$ \\
            We can therefore claim that $n^7 \equiv 36n + 7n \pmod{42}$ and we prove this below by showing that this solution satisfies both of the above equations:
            \begin{enumerate}
                \item $n^7 \equiv (36n + 7n) \equiv 1n + 0 \equiv n \pmod{7}$
                \item $n^7 \equiv (36n + 7n) \equiv 0 + 1n \equiv n \pmod{6}$
            \end{enumerate}
            Therefore, $n^7 \equiv 43n \equiv n \pmod{42}$ as required.
    \end{enumerate}

    \section{2.3}
    \begin{enumerate}
        \item \rtp $\myforall{n} ((\myexists{i,j} n=i^2-j^2) \iff (n\equiv 0\pmod{4} \lor n\equiv 1\pmod{4} \lor n\equiv 3\pmod{4}))$
        Let $n$ arbitrary integer. \\
        First we prove `$\Leftarrow$'.\\
        \rtp $\myexists{i,j} n=i^2-j^2$ \\ 
        Note that the following cases are exhaustive but not mutually exclusive.
        \begin{enumerate}
            \case{0} $n \equiv 0 \pmod 4$ \\
                $\therefore n=4a$ for some integer $a$ \\
                $\therefore n=(a+1)^2 - (a-1)^2$
            \case{1} $n \equiv 1 \pmod 4$ \\
                $\therefore n=4a + 1$ for some integer $a$ \\
                $\therefore n=(2a+1)^2 - (2a)^2$
            \case{2} $n \equiv 3 \pmod 4$ \\
                $\therefore n=4a + 3$ for some integer $a$ \\
                $\therefore n=(2a+2)^2 - (2a+1)^2$
        \end{enumerate}
        Now we prove `$\Rightarrow$' \\
        Assume $\myexists{i,j} n=i^2-j^2$ \\
        Let $i, j$ such that $n=i^2-j^2=(i-j)(i+j)$ \\
        \rtp $n\equiv 0\pmod{4} \lor n\equiv 1\pmod{4} \lor n\equiv 3\pmod{4}$
        \begin{enumerate}
            \case{0} $i$ is odd and $j$ is odd \\
                Therefore $i-j = 2a, i+j = 2b$ for some integers $a, b$ \\
                Therefore $n=(i-j)(i+j)=4ab\equiv 0 \pmod{4}$
            \case{1} Exactly one of $i$ and $j$ is even. Without loss of generality, take $i$ is odd and $j$ is even. \\
                Therefore $i-j = 2a+1, i+j = 2b+1$ for some integers $a, b$ \\
                Therefore $n=(i-j)(i+j) = 4ab + 2(a+b) + 1 \equiv 2c + 1 \pmod{4}$ where $c=a+b$ \\
                Therefore $n\equiv 1 \pmod{4} \lor n\equiv 3\pmod{4}$
            \case{2} $i$ is even and $j$ is even \\
                Therefore $i-j = 2a, i+j = 2b$ for some integers $a, b$ \\
                Therefore $n=(i-j)(i+j)=4ab\equiv 0 \pmod{4}$
        \end{enumerate}
    \item $ $
        \begin{enumerate}
            \item 1, 11, 111\\
                1, 3, 7
            \item The $k^{\textrm{th}}$ decimal repunit in base $n$ can be written as $\frac{n^k-1}{n-1}$ \\
                Consider the expression $(2a)^k -1 \pmod(4)$ in the two following exhaustive cases
                \begin{enumerate}
                    \case{0} $k2i$ for some integer $i$ \\
                        \begin{align*}
                            (2a)^k -1 &\equiv 4^i\cdot a^k - 1 \\
                            &\equiv -1 \\
                            &\equiv 3 \pmod{4}
                        \end{align*}
                    \case{1} $k=2i+1$ for some integer $i$ \\
                        \begin{align*}
                            (2a)^l -1 &\equiv 4^i\cdot 2\cdot a^k -1 \\
                            &\equiv -1 \\
                            &\equiv 3 \pmod{4}
                        \end{align*}
                \end{enumerate}
                As such, the expression is always congruent to $3 \pmod{4}$. \\
                Next, note that $n-1$ is a square number $\Rightarrow (\frac{n^k -1}{n-1}$ is a square number $\Rightarrow n^k-1$ is a square number$)$ \\
                Therefore, for all bases $n$ such that $n$ is even and $n-1$ is square (for example, $n=2$ or $n=10$), then $\frac{n^k-1}{n-1}\equiv 3 \pmod{4}$, which, by Lemma 26, means it cannot be a square number.
        \end{enumerate}
        
    \end{enumerate}

\end{document}